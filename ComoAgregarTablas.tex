

AGREGAR TABLAS\par
{
\extrarowheight = -0.3ex
\renewcommand{\arraystretch}{1.8}

\begin{table}[H]
    \centering
    \caption{} \vspace{5mm}
    \textbf{Tabla con Títulos en las columnas} \vspace{7mm}
   
        \begin{tabular}{cccc}
        \hline
        Titulo1 & Titulo2 & Titulo3 & Titulo4  \\
        \hline
            150        & 210       & 130        & 1 \\
            1        & 1        & 1        & 1 \\
            1        & 1        & 1        & 1 \\
        \hline
        \end{tabular}
    \label{TablaI}
\end{table}}
\setlength{\parskip}{-1mm}
Nota: Descripción del contenido de la tabla. [ ] % Coloque el texto "Nota:" y luego describa la tabla, Si es tomada o elaborada con datos de otro autor coloque el número correspondiente
\setlength{\parskip}{2em}

{
\extrarowheight = -0.3ex % Agrega separación entre las filas
\renewcommand{\arraystretch}{1.8} % Agrega separación entre las filas
\begin{table}[H]
    \centering
    \caption{} \vspace{5mm}
    \textbf{Tabla con Títulos en las columnas y las filas} \vspace{7mm}
   
        \begin{tabular}{ccccc}
\hline
            & Titulo1 & Titulo2 & Titulo3 & Titulo4  \\
\hline
    Fila1        & 1        & 1        & 1        & 1 \\
    Fila2        & 1        & 1        & 1        & 1 \\
    Fila3        & 1        & 1        & 1        & 1 \\
\hline
        \end{tabular}
    \label{TablaII}
\end{table}}
\setlength{\parskip}{-1mm}
Nota: Descripción del contenido de la tabla. [ ] % Coloque el texto "Nota:" y luego describa la tabla, Si es tomada o elaborada con datos de otro autor coloque el número correspondiente
\setlength{\parskip}{2em}

{
\extrarowheight = -0.3ex
\renewcommand{\arraystretch}{1.8}
\begin{table}[H]
    \centering
    \caption{} \vspace{5mm}
    \textbf{Tabla con Celdas combinadas, títulos en las columnas y las filas} \vspace{7mm}
   
        \begin{tabular}{ccccc}
\hline
& \multicolumn{4}{c}{MES1}\\
             & Semana1 & Semana2 & Semana3 & Semana4  \\
\hline
    Fila1        & 1        & 1        & 1        & 1 \\
    Fila2        & 1        & 1        & 1        & 1 \\
    Fila3        & 1        & 1        & 1        & 1 \\
\hline
        \end{tabular}
    \label{TablaIII}
\end{table}}
\setlength{\parskip}{-1mm}
Nota: Descripción del contenido de la tabla. [ ] % Coloque el texto "Nota:" y luego describa la tabla, Si es tomada o elaborada con datos de otro autor coloque el número correspondiente
\setlength{\parskip}{2em}

{
\extrarowheight = -0.3ex
\renewcommand{\arraystretch}{1.8}
\begin{table}[H]
    \centering
    \caption{} \vspace{5mm}
    \textbf{Tabla con formulas matemáticas} \vspace{7mm}
   
    \begin{tabular}{>{$}c<{$}>{$}c<{$}>{$}c<{$}>{$}c<{$}>{$}c<{$}}
        \hline
        & \multicolumn{4}{c}{MES1}\\
                     & Semana1 & Semana2 & Semana3 & Semana4  \\
        \hline
            Fila1        & 1        & 1        & 1        & 1 \\
            Fila2        & 1        & 1        & 1        & 1 \\
            Fila3        & 1        & 1        & 1        & 1 \\
        \hline
    \end{tabular}
    \label{TablaIV}
\end{table}}
\setlength{\parskip}{-1mm}
Nota: Descripción del contenido de la tabla. [ ] % Coloque el texto "Nota:" y luego describa la tabla, Si es tomada o elaborada con datos de otro autor coloque el número correspondiente
\setlength{\parskip}{2em}

{
\extrarowheight = -0.3ex
\renewcommand{\arraystretch}{1.8}
\begin{table}[H]
    \centering
    \caption{} \vspace{5mm}
    \textbf{Tabla con formulas matemáticas} \vspace{7mm}
   
    \begin{tabular}{lUUUU}
        \hline
        & \multicolumn{4}{c}{MES1}\\
                     & Semana1 & Semana2 & Semana3 & Semana4  \\
        \hline
            Fila1        & 1        & 1        & 1        & 1 \\
            Fila2        & 1        & 1        & 1        & 1 \\
            Fila3        & 1        & 1        & 1        & 1 \\
        \hline
    \end{tabular}
    \label{TablaV}
\end{table}}
\setlength{\parskip}{-1mm}
Nota: Descripción del contenido de la tabla. [ ] % Coloque el texto "Nota:" y luego describa la tabla, Si es tomada o elaborada con datos de otro autor coloque el número correspondiente
\setlength{\parskip}{2em}

{
\extrarowheight = -0.3ex
\renewcommand{\arraystretch}{1.8}
\begin{table}[H]
    \centering
    \caption{} \vspace{5mm}
    \textbf{Tabla con imágenes y títulos en las columnas y las filas} \vspace{7mm}
   
    \begin{tabular}{>{\arraybackslash}m{2cm}>{\centering\arraybackslash}m{2cm}>{\centering\arraybackslash}m{2cm}>{\centering\arraybackslash}m{2cm}>{\centering\arraybackslash}m{3cm}}
\hline
            & Titulo1 & Titulo2 & Titulo3 & Titulo4  \\
\hline
    Fila1       & 1        & 1        & 1        & \includegraphics[scale=0.2]{Figuras/libro} \\
    Fila2       & 1        & 1        & 1        & \includegraphics[scale=0.2]{Figuras/TG} \\
    Fila3       & 1        & 1        & 1        & \includegraphics[scale=0.2]{Figuras/BD} \\
\hline
        \end{tabular}
    \label{TablaVI}
\end{table}}
\setlength{\parskip}{-1mm}
Nota: Descripción del contenido de la tabla. [ ] % Coloque el texto "Nota:" y luego describa la tabla, Si es tomada o elaborada con datos de otro autor coloque el número correspondiente
\setlength{\parskip}{2em}


TABLA QUE OCUPA MAS DE UNA PÁGINA

{\extrarowheight = -0.3ex
\renewcommand{\arraystretch}{1.8}
\begin{longtable}{p{2in}cc}
\caption{Un ejemplo sencillo}\\
\hline\hline
\multicolumn{3}{c}{\textbf{Tabla repetitiva}}\\
\hline\hline
\multicolumn{2}{c}{Primera y segunda columnas} & {Tercera columna}\\
\hline
\endfirsthead
\caption[]{(Continuación)}\\
\hline
\multicolumn{2}{c}{Primera y segunda columnas} & {Tercera columna}\\
\hline
Subtitulo1 & Subtitulo2 & Subtitulo3 \\
\hline
\endhead
\hline
%\multicolumn{3}{|c|}{Sigue $\ldots$}\\
%\hline
\endfoot
\hline
\endlastfoot
Subtitulo1 & Subtitulo2 & Subtitulo3 \\
\hline\hline
Muchas líneas como & ésta & 2 \\
Muchas líneas como & ésta & 2 \\
Muchas líneas como & ésta & 2 \\
Muchas líneas como & ésta & 2 \\
Muchas líneas como & ésta & 2 \\
Muchas líneas como & ésta & 2 \\
Muchas líneas como & ésta & 2 \\
Muchas líneas como & ésta & 2 \\
Muchas líneas como & ésta & 2 \\
Muchas líneas como & ésta & 2 \\
Muchas líneas como & ésta & 2 \\
Muchas líneas como & ésta & 2 \\
Muchas líneas como & ésta & 2 \\
Muchas líneas como & ésta & 2 \\
Muchas líneas como & ésta & 2 \\
Muchas líneas como & ésta & 2 \\
Muchas líneas como & ésta & 2 \\
Muchas líneas como & ésta & 2 \\
Muchas líneas como & ésta & 2 \\
Muchas líneas como & ésta & 2 \\
Muchas líneas como & ésta & 2 \\
Muchas líneas como & ésta & 2 \\
Muchas líneas como & ésta & 2 \\
Muchas líneas como & ésta & 2 \\
Muchas líneas como & ésta & 2 \\
Muchas líneas como & ésta & 2 \\
Muchas líneas como & ésta & 2 \\
Muchas líneas como & ésta & 2 \\
Muchas líneas como & ésta & 2 \\
Muchas líneas como & ésta & 2 \\
Muchas líneas como & ésta & 2 \\
Muchas líneas como & ésta & 2 \\
Muchas líneas como & ésta & 2 \\
Muchas líneas como & ésta & 2 \\
Muchas líneas como & ésta & 2 \\
Muchas líneas como & ésta & 2 \\
Muchas líneas como & ésta & 2 \\
Muchas líneas como & ésta & 2 \\
\end{longtable}}
\setlength{\parskip}{-1mm}
Nota: Descripción del contenido de la tabla. [ ] % Coloque el texto "Nota:" y luego describa la tabla, Si es tomada o elaborada con datos de otro autor coloque el número correspondiente
\setlength{\parskip}{2em}
